% Options for packages loaded elsewhere
\PassOptionsToPackage{unicode}{hyperref}
\PassOptionsToPackage{hyphens}{url}
\PassOptionsToPackage{dvipsnames,svgnames,x11names}{xcolor}
%
\documentclass[
  10pt,
  a4paper,
]{scrreprt}

\usepackage{amsmath,amssymb}
\usepackage{iftex}
\ifPDFTeX
  \usepackage[T1]{fontenc}
  \usepackage[utf8]{inputenc}
  \usepackage{textcomp} % provide euro and other symbols
\else % if luatex or xetex
  \usepackage{unicode-math}
  \defaultfontfeatures{Scale=MatchLowercase}
  \defaultfontfeatures[\rmfamily]{Ligatures=TeX,Scale=1}
\fi
\usepackage{lmodern}
\ifPDFTeX\else  
    % xetex/luatex font selection
\fi
% Use upquote if available, for straight quotes in verbatim environments
\IfFileExists{upquote.sty}{\usepackage{upquote}}{}
\IfFileExists{microtype.sty}{% use microtype if available
  \usepackage[]{microtype}
  \UseMicrotypeSet[protrusion]{basicmath} % disable protrusion for tt fonts
}{}
\makeatletter
\@ifundefined{KOMAClassName}{% if non-KOMA class
  \IfFileExists{parskip.sty}{%
    \usepackage{parskip}
  }{% else
    \setlength{\parindent}{0pt}
    \setlength{\parskip}{6pt plus 2pt minus 1pt}}
}{% if KOMA class
  \KOMAoptions{parskip=half}}
\makeatother
\usepackage{xcolor}
\usepackage[inner=2cm,outer=2cm,top=2cm,bottom=2cm,headsep=22pt,headheight=11pt,footskip=33pt,ignorehead,ignorefoot,heightrounded]{geometry}
\setlength{\emergencystretch}{3em} % prevent overfull lines
\setcounter{secnumdepth}{1}
% Make \paragraph and \subparagraph free-standing
\ifx\paragraph\undefined\else
  \let\oldparagraph\paragraph
  \renewcommand{\paragraph}[1]{\oldparagraph{#1}\mbox{}}
\fi
\ifx\subparagraph\undefined\else
  \let\oldsubparagraph\subparagraph
  \renewcommand{\subparagraph}[1]{\oldsubparagraph{#1}\mbox{}}
\fi


\providecommand{\tightlist}{%
  \setlength{\itemsep}{0pt}\setlength{\parskip}{0pt}}\usepackage{longtable,booktabs,array}
\usepackage{calc} % for calculating minipage widths
% Correct order of tables after \paragraph or \subparagraph
\usepackage{etoolbox}
\makeatletter
\patchcmd\longtable{\par}{\if@noskipsec\mbox{}\fi\par}{}{}
\makeatother
% Allow footnotes in longtable head/foot
\IfFileExists{footnotehyper.sty}{\usepackage{footnotehyper}}{\usepackage{footnote}}
\makesavenoteenv{longtable}
\usepackage{graphicx}
\makeatletter
\def\maxwidth{\ifdim\Gin@nat@width>\linewidth\linewidth\else\Gin@nat@width\fi}
\def\maxheight{\ifdim\Gin@nat@height>\textheight\textheight\else\Gin@nat@height\fi}
\makeatother
% Scale images if necessary, so that they will not overflow the page
% margins by default, and it is still possible to overwrite the defaults
% using explicit options in \includegraphics[width, height, ...]{}
\setkeys{Gin}{width=\maxwidth,height=\maxheight,keepaspectratio}
% Set default figure placement to htbp
\makeatletter
\def\fps@figure{htbp}
\makeatother
\newlength{\cslhangindent}
\setlength{\cslhangindent}{1.5em}
\newlength{\csllabelwidth}
\setlength{\csllabelwidth}{3em}
\newlength{\cslentryspacingunit} % times entry-spacing
\setlength{\cslentryspacingunit}{\parskip}
\newenvironment{CSLReferences}[2] % #1 hanging-ident, #2 entry spacing
 {% don't indent paragraphs
  \setlength{\parindent}{0pt}
  % turn on hanging indent if param 1 is 1
  \ifodd #1
  \let\oldpar\par
  \def\par{\hangindent=\cslhangindent\oldpar}
  \fi
  % set entry spacing
  \setlength{\parskip}{#2\cslentryspacingunit}
 }%
 {}
\usepackage{calc}
\newcommand{\CSLBlock}[1]{#1\hfill\break}
\newcommand{\CSLLeftMargin}[1]{\parbox[t]{\csllabelwidth}{#1}}
\newcommand{\CSLRightInline}[1]{\parbox[t]{\linewidth - \csllabelwidth}{#1}\break}
\newcommand{\CSLIndent}[1]{\hspace{\cslhangindent}#1}

\addtokomafont{disposition}{\rmfamily}
\usepackage{ragged2e}
\usepackage{blindtext}\usepackage{amsthm}
\usepackage{amsthm}
\usepackage{hyperref}
\newtheorem{thm}{Theorem}[subsection]
\renewcommand{\thethm}{\arabic{subsection}.\arabic{thm}}

\makeatletter
\makeatother
\makeatletter
\makeatother
\makeatletter
\@ifpackageloaded{caption}{}{\usepackage{caption}}
\AtBeginDocument{%
\ifdefined\contentsname
  \renewcommand*\contentsname{Table of contents}
\else
  \newcommand\contentsname{Table of contents}
\fi
\ifdefined\listfigurename
  \renewcommand*\listfigurename{List of Figures}
\else
  \newcommand\listfigurename{List of Figures}
\fi
\ifdefined\listtablename
  \renewcommand*\listtablename{List of Tables}
\else
  \newcommand\listtablename{List of Tables}
\fi
\ifdefined\figurename
  \renewcommand*\figurename{Figure}
\else
  \newcommand\figurename{Figure}
\fi
\ifdefined\tablename
  \renewcommand*\tablename{Table}
\else
  \newcommand\tablename{Table}
\fi
}
\@ifpackageloaded{float}{}{\usepackage{float}}
\floatstyle{ruled}
\@ifundefined{c@chapter}{\newfloat{codelisting}{h}{lop}}{\newfloat{codelisting}{h}{lop}[chapter]}
\floatname{codelisting}{Listing}
\newcommand*\listoflistings{\listof{codelisting}{List of Listings}}
\usepackage{amsthm}
\theoremstyle{definition}
\newtheorem{definition}{Definition}[section]
\theoremstyle{plain}
\newtheorem{theorem}{Theorem}[section]
\theoremstyle{remark}
\AtBeginDocument{\renewcommand*{\proofname}{Proof}}
\newtheorem*{remark}{Remark}
\newtheorem*{solution}{Solution}
\makeatother
\makeatletter
\@ifpackageloaded{caption}{}{\usepackage{caption}}
\@ifpackageloaded{subcaption}{}{\usepackage{subcaption}}
\makeatother
\makeatletter
\@ifpackageloaded{tcolorbox}{}{\usepackage[skins,breakable]{tcolorbox}}
\makeatother
\makeatletter
\@ifundefined{shadecolor}{\definecolor{shadecolor}{rgb}{.97, .97, .97}}
\makeatother
\makeatletter
\makeatother
\makeatletter
\makeatother
\ifLuaTeX
  \usepackage{selnolig}  % disable illegal ligatures
\fi
\IfFileExists{bookmark.sty}{\usepackage{bookmark}}{\usepackage{hyperref}}
\IfFileExists{xurl.sty}{\usepackage{xurl}}{} % add URL line breaks if available
\urlstyle{same} % disable monospaced font for URLs
\hypersetup{
  pdftitle={Annual Progress Review},
  pdfauthor={Thomas William Boughen},
  colorlinks=true,
  linkcolor={blue},
  filecolor={Maroon},
  citecolor={Blue},
  urlcolor={Blue},
  pdfcreator={LaTeX via pandoc}}

\title{Annual Progress Review}
\author{Thomas William Boughen}
\date{}

\begin{document}
\cleardoublepage
\thispagestyle{empty}
{\centering
\hbox{}\vskip 0cm plus 1fill
{\Huge\bfseries Annual Progress Review \par}
\vspace{12ex}
{\Large\bfseries Thomas William Boughen \par}
\vspace{3ex}
\vskip 0cm plus 2fill
%{\bfseries\large Doctor of Philosophy \par}
\vspace{3ex}
{\bfseries\large  \par}
\vspace{12ex}
{\includegraphics[width=0.1\linewidth]{"imgs/University_of_Newcastle_Coat_of_Arms.png"}\par}
%
%
{\bfseries\large Newcastle University \par}
\vspace{3ex}
%
{\bfseries\large School of Mathematics, Statistics and Physics \par}
%
\vspace{12ex}
%{\small Submitted in total fulfilment of the requirements
%of the degree of Doctor of Philosophy \par}
%}
\footnote{An html version of this report can be found at \url{www.twboughen.github.io/phd/APR/report}}
\justifying
\noindent\ifdefined\Shaded\renewenvironment{Shaded}{\begin{tcolorbox}[boxrule=0pt, interior hidden, frame hidden, enhanced, breakable, sharp corners, borderline west={3pt}{0pt}{shadecolor}]}{\end{tcolorbox}}\fi

\hypertarget{sec-int}{%
\chapter{Introduction}\label{sec-int}}

Since the aim is to gain understanding about the behaviour of the degree
distribution of networks at the right tail, it seems natural to look to
using methods from extreme value theory.

\hypertarget{sec-ext}{%
\chapter{Extreme Value Theory}\label{sec-ext}}

This section begins with a review of the theory and methodology for
modelling the extreme values of continuous random variables, before
moving to considerations for modelling the extreme values of discrete
random variables.

\hypertarget{sec-ce}{%
\section{Continuous Extremes}\label{sec-ce}}

Studying the properties of the extreme values of a random variable first
requires determining what exactly is considered to be an extreme value.
In this section extreme values of two kinds are considered, both of
which can be characterised.

The first kind of extreme value considers the distribution of block
maxima. That is, for a set of independent and identically distributed
(iid) random variables \(X_1,\ldots,X_n\) with common cumulative density
function (cdf) \(F\) what is the limiting distribution of
\(M_n = \max\{X_1,\ldots,X_n\}\)?

Clearly, as \(n\rightarrow \infty\), the block maxima \(M_n\) converges
almost surely to the right endpoint of \(F\). However, standardising the
block maxima allows for some characterisation of the limiting
distribution.

\begin{theorem}[Fisher--Tippett--Gnedenko
Theorem]\protect\hypertarget{thm-evt}{}\label{thm-evt}

With \(X_1, \ldots,X_n \overset{\mathrm{iid}}{\sim} F\) and
\(\{ a_n\}_{n\ge0}, \{ b_n\}_{n\ge0}\) such that:

\[\lim_{n\rightarrow\infty}\Pr\left(\frac{1}{a_n}[M_n-b_n]\le x\right) = G(x),\]
for some non-degenerate \(G\).

Then \(F\) is said to be in the (maximum) domain of attraction of \(G\),
denoted \(F\in\mathcal D(G)\) ,and \(G\) is of one of three types:

\begin{itemize}
\tightlist
\item
  Gumbel: \(\Lambda(x) = \exp\{-\exp(-x)\},\quad x \in \mathbb R\)
\item
  Fréchet:
  \(\Phi_\alpha(x) = \exp\{-x^{-\alpha}\},\quad x\ge 0,\alpha>0\)
\item
  Negative-Weibull:
  \(\Psi_\alpha(x) = \exp\{-x^{-a}\},\quad x<0,\alpha>0\)
\end{itemize}

\end{theorem}

Each of these three types defines a domain of attraction.

\begin{definition}[Domains of
Attraction]\protect\hypertarget{def-doa}{}\label{def-doa}

The three domains of attraction that result from Theorem~\ref{thm-evt}
have the following equivalent conditions:

For a distribution with cdf \(F\) and survival function \(\bar F\) that
has right endpoint \(x_F\), the distribution belongs to each domain of
attraction subject to the conditions below:

\textbf{If \(x_F=\infty\): }

\begin{itemize}
\tightlist
\item
  Type I/Gumbel/\(\mathcal D(\Lambda)\): \[
  \lim_{x\rightarrow\infty} \frac{\bar F(x+ta(x))}{\bar F(x)} = e^{-t},\quad \forall t>0 
  \]
\item
  Type II/Fréchet/\(\mathcal D (\Phi_\alpha)\): \[
  \lim_{x\rightarrow\infty} \frac{\bar F(tx)}{\bar F(x)} = x^{-\alpha}, \quad \forall t>0 \quad \text{ for some } \alpha>0
  \]
\end{itemize}

\textbf{If \(x_F<0\):}

\begin{itemize}
\tightlist
\item
  Type III/Negative-Weibull/\(\mathcal D(\Psi_\alpha)\): \[
  \lim_{h\downarrow 0}\frac{\bar F(x_F-xh)}{\bar F(x_F-h)} = x^\alpha, \quad\alpha>0
  \]
\end{itemize}

\end{definition}

The parameter \(\alpha\) in Definition~\ref{def-doa} and
Theorem~\ref{thm-evt} is called the extreme value index.

Here, distributions in the Gumbel domain are referred to as light
tailed, distributions in the Negative-Weibull domain are referred to as
short tailed, and those in the Fréchet are referred to as heavy
tailed.This terminology for heavy tailed distributions in different ot
some of the literature that defined a heavy tailed distribution as one
that decays slower than exponential. However the terminology used here
is also widely used.

Throughout this report functions will be referred to as regularly
varying or slowly varying, what is meant by this is formally deined
below:

\begin{definition}[Regular
Variation]\protect\hypertarget{def-rv}{}\label{def-rv}

A positive,real valued, measurable function \(f\) is said to be
regularly varying at infinity with index \(\gamma\) if for all \(t>0\):

\[
\lim_{x\rightarrow\infty}\frac{f(tx)}{f(x)} = x^{\gamma}.
\] If \(\gamma =0\), then \(f\) is instead said to be slowly varying.

\end{definition}

Note that the condition for a distribution to belong to the Fréchet
domain is equivalent to saying that the survival function \(\bar F\) is
regularly varying with index \(-\alpha\).

In addition to heavy tailed distributions it is also useful to define
what will be referred to as super heavy tailed distributions. This term
is often just refers to specific distributions such as the log-Cauchy
distriubtion, but Fraga Alves, Haan, and Neves (2009) provides the more
precise definition below:

\begin{definition}[Super Heavy
Tails]\protect\hypertarget{def-sup}{}\label{def-sup}

A distribution is with survival function \(\bar F\) is said to have
super heavy tails if: \[
\lim_{x\rightarrow\infty}\frac{\bar F(tx)}{\bar F (x)} = 1,\qquad \forall t>0
\] That is, a distribution is called super heavy if its survival
function is slowly varying.

\end{definition}

The three main types of extremal distribution (Gumbel, Fréchet and
Negative-Weibull) can be united into one distribution, called the
Generalised Extreme Value (GEV) distribution.

\begin{definition}[Generalised Extreme Value
Distribution]\protect\hypertarget{def-gev}{}\label{def-gev}

Denoted by \(\text{GEV}(\mu,\sigma,\xi)\) the distribution is
characterised by three parameters \(\mu \in \mathbb R\) the location,
\(\sigma\in \mathbb R^+\) the scale, and the shape \(\xi\in \mathbb R\).
It has support on \(\{x\in \mathbb R:1+\xi(x-\mu)/\sigma > 0\}\) and has
cdf given by:

\[
G(x) = \begin{cases}\exp\left\{-\left(1+\displaystyle\frac{\xi(x-\mu)}{\sigma}\right)_+^{-1/\xi}\right\},&\xi\ne0\\
\exp\left\{-\exp\left(-\displaystyle\frac{x-\mu}{\sigma}\right)\right\},&\xi=0.
\end{cases}
\]

\end{definition}

The three types of extremal distribution are obtained from changing the
shape parameter \(\xi\), which corresponds to \(1/\alpha\) in
Theorem~\ref{thm-evt}. This change is generally to made so that
increasing \(\xi\) corresponds to increasing how heavy the tails of the
distribution are. Specifically, \(\xi<0\), \(\xi=0\), \(\xi>0\),
correspond to the negative Weibull, Gumbel and the Fréchet domains of
attraction respectively.

Another kind of extreme values are the observations above a large
threshold, like the limiting distribution of block maxima, the limiting
distribution of these extreme values can be characterised by the
generalised pareto (GP) distribution.

\begin{definition}[Generalised Pareto
Distribution]\protect\hypertarget{def-gp}{}\label{def-gp}

Consider a random variable \(X\) with the same cdf \(F\) as in
Theorem~\ref{thm-evt}, the Generalised Pareto (GP) distribution can be
obtained by using the GEV distribution and conditional probability such
that for large enough threshold the GP distribution approximately
describes the conditional distribution of threshold exceedances. More
precisely, for sufficiently large threshold \(u\) and the change of
variable to \(Y=X-u\): \[
\Pr(Y\le y | Y>0) = H(y) = \begin{cases}
1-\left(1+\displaystyle\frac{\xi y}{\sigma}\right)^{-1/\xi},&y>0,\xi\ne 0 \\
1-\exp\left(-\displaystyle\frac{y}{\sigma}\right),&y>0,\xi = 0
\end{cases}
\]

\end{definition}

Since this distribution was obtained using a
\(\text{GEV}(\mu,\sigma^*,\xi)\) the shape parameter \(\xi\) is
identical in both distributions and the shape parameter \(\sigma\) is
defined such that \(\sigma = \sigma^* + \xi(u-\mu)\).

It is also possible to derive the result without using the GEV, as shown
in {[}REF{]}.

\hypertarget{discrete-extremes}{%
\section{Discrete Extremes}\label{discrete-extremes}}

A lot of Section~\ref{sec-ce} is appropriate only for continuous random
variables and some of the results may not hold in a discrete setting. In
particular, a continuous distribution \(F\) being in certain domain of
attraction may not necessarily imply that a discretisation of \(F\)
remains in that domain of attraction.

\begin{definition}[Discretisation]\protect\hypertarget{def-disc}{}\label{def-disc}

The discretisation of a distribution with cdf \(F\) is given by

\[F^*(n) = F(n) - F(n-1), \quad n   \in \mathbb Z\]

\end{definition}

Shimura (2012) provides conditions for a discretisation of a continuous
distribution to belong to the same domain of attraction. In particular
the following theorem which corresponds to Theorem 1 in Shimura (2012).

\begin{theorem}[Domain of attraction
consistency]\protect\hypertarget{thm-shimura1}{}\label{thm-shimura1}

~

\begin{enumerate}
\def\labelenumi{(\alph{enumi})}
\tightlist
\item
  Every discretisation of distribution in \(\mathcal D(\Phi_\alpha)\)
  remains in \(\mathcal D(\Phi_\alpha)\).
\item
  The discretisation of a distribution remains in
  \(\mathcal D(\Lambda)\) if and only if the original is in
  \(\mathcal D(\Lambda)\cap \mathcal L\).
\end{enumerate}

Where \(\mathcal L\) is the set of long-tailed distributions that have
the property: \[
\lim_{x\rightarrow \infty}\frac{\overline F(x+1)}{\overline F(x)} = 1   
\]

\end{theorem}

In addition Shimura (2012) introduces a quantity useful for determining
the domain of a attraction that a discrete distribution belongs to.

\begin{definition}[Omega
Function]\protect\hypertarget{def-omega}{}\label{def-omega}

For a distribution \(F\) with survival function \(\overline F\) and some
\(n\in\mathbb Z^+\) let:

\[
\Omega(F,n) = \left(\log\frac{\overline F (n+1)}{\overline F (n+2)}\right)^{-1} - \left(\log\frac{\overline F (n)}{\overline F (n+1)}\right)^{-1}
\]

\end{definition}

This quantity plays an important role in Section~\ref{sec-meth} when
determining the domain of attraction to which the degree distribution of
a network generative model belongs.

Applying ideas from Section~\ref{sec-ce} to modelling discrete random
variables has been approached from many different directions. What
follows is a overview of some of the approaches that have been taken but
will see use in this report.

Hitz, Davis, and Samorodnitsky (2024) note that using the GP
distribution as an approximation in a discrete setting leads to bias in
the likelihood function and can lead to it being inadequate for
modelling. They propose two other peaks over threshold methods that rely
on parametric families of discrete distributions. The first, what they
refer to as the discrete generalised Pareto approximation is based on an
extension of the discrete survival function. The second, the generalised
Zipf distribution is obtained from an extension of the probability mass
function. Both methods are motivated theoretically for modelling of a
large class of discrete distributions and are shown in the paper to
either match or outperform using the GP to model discrete data directly.

Ahmad, Gaetan, and Naveau (2022) first introduce an extended GP
distribution, a continuous distribution that extends the idea of
obtaining GP values from a probability integral transform (PIT) of
\(U(0,1)\) draws and instead considers a PIT of draws from any
distribution on \((0,1)\) such as a beta distribution. This distribution
is then discretised into their discrete extended GP distribution.

\hypertarget{sec-mod}{%
\section{Modelling}\label{sec-mod}}

The results from Section~\ref{sec-ce} allow the GEV and GP to be fitted
to the block maxima and exceedances respectively. Typically, when
fitting the GP, a sufficiently high threshold needs to be specified
beforehand {[}REFER TO COLES 2001{]} and give examples.

Another more recent approach shown in {[}MACDONALD 2012{]}, uses a
spliced threshold mixture to model the threshold exceedances where one
distribution is assumed for the bulk of the data and the GP is used for
those values above the threshold. This approach can also be applied in
the discrete setting, and is what is used in Section~\ref{sec-meth}.

\hypertarget{networks}{%
\chapter{Networks}\label{networks}}

Networks are the structures that will be the source of data that the
results from Section~\ref{sec-ext} will be used to analyse. Networks
appear across a wide range of fields when attempting to represent
complex systems and the relationships between the components within,
showing up in anything from micro-biology (e.g.~protein interactions in
cells) to sociology (e.g.~the social network of Harvard graduates).

This makes networks a valuable source of data and understanding the
mechanics of the network generation process can provide insights to the
components themselves and into the networks future.

\hypertarget{mathematical-definitions}{%
\section{Mathematical Definitions}\label{mathematical-definitions}}

Networks on the face of it are fairly simple objects, nothing more than
a collection of objects with connections between each other. Here,
graphs constructed from vertices and edges will be used as an analogue
for these networks.

\begin{definition}[Graph/Network]\protect\hypertarget{def-net}{}\label{def-net}

A graph \(G = (V,E)\) is constructed from a vertex set
\(V\in\mathbb Z^+\) and an edge set \(E\). The edge set can take on one
of two forms depending on if the graph is directed or un-directed. If
the graph is directed then \(E\subseteq V^2\) i.e the edge set is
contained within the set of ordered pairs of vertices, whereas if the
graph is \textbf{un-directed} then \(E\subseteq [V]^2\) i.e.~the edge
set is contained within the set of un-ordered pairs of vertices. The
focus from now on will be on un-directed networks and graphs.

\end{definition}

Throughout this section and the next the concept of a vertices
``degree'' will come up, and in fact the main focus of
Section~\ref{sec-meth} is the degree distribution of networks.

\begin{definition}[Degree]\protect\hypertarget{def-deg}{}\label{def-deg}

For an un-directed graph a vertex's degree denoted \(d(v)\) or \(k_v\)
for \(v\in V\) is the number of edges that are connected to vertex
\(v\): \[
d(v) = |\{e\in E : v \in e\}|
\] Directed graphs have something analogous, called the in-degree
\(d_{in}\), out-degree \(d_{out}\) and total degree \(d_{tot}\). The
in-degree of a vertex \(v\) is the number edges with endpoint at \(v\),
whereas the out-degree is the number of edges with start point at \(v\)
and the total degree is the sum of these i.e.:

\begin{align*}
d_{in}(v)&= |\{(w_1,w_2)\in E: w_2=v \}|\\
d_{out}(v) &= |\{(w_1,w_2)\in E: w_1=v \}|\\
d_{tot}(v) &= d_{in}(v) + d_{out}(v)
\end{align*}

\end{definition}

There are many reasons to analyse network like data, one of which is to
gain an insight into the mechanics that governed the growth of the
network. The next sub-section is focused on presenting several network
generative models increasing in generality.

\hypertarget{network-generative-models}{%
\section{Network Generative Models}\label{network-generative-models}}

The models in this section begin with the most general considered in
this report, and then two special cases of this model are introduced.

\hypertarget{general-preferential-attachment-gpa}{%
\subsection{General Preferential Attachment
(GPA)}\label{general-preferential-attachment-gpa}}

Under this model, at each time step one vertex is added to the network
and brings an edge with it that connects the existing vertices with a
probability proportional to some function of the vertices' degrees.

\begin{definition}[General Preferential Attachment
Model]\protect\hypertarget{def-gpa}{}\label{def-gpa}

Starting with a graph
\(G_1 = (V_1, E_1) = (\{1,\ldots,m_0\}, \emptyset)\), at each following
time step \(t>1\) the graph is denoted by \(G_t = (V_t, E_t)\) and is
generated by repeating:

\begin{enumerate}
\def\labelenumi{\arabic{enumi}.}
\tightlist
\item
  \textbf{Growth:} Add a new vertex to the vertex set i.e. \[
  V_t = V_{t-1} \cup \{t\}
  \]
\item
  \textbf{Preferential Attachment:} Add \(m\le m_0\) edges connecting
  the new vertex to those already in the graph selected at random
  proportional to a function of their degree(minus one)\footnote{The
    probabilities are proportional to the degree minus one to align with
    the results from {[}GPA REF{]}} i.e.: \[
  E_t  = E_{t-1} \cup \tilde E
  \] where \(\tilde E = \{\tilde e_1,\ldots, \tilde e_m\}\) and
  \(\tilde e_i = \{t,\tilde v_i\}\) for
  \(\tilde v_i \sim \text{Cat}(V_{t-1}, P)\)
\end{enumerate}

\begin{align*}
P &= \left\{\frac{g(k_v-1)}{\sum_{w\in V_{t-1}} g(k_w-1)} : v \in V_{t-1}\right\}
\end{align*}

for some function \(g: \mathbb Z \mapsto \mathbb R^+\setminus\{0\}\),
which will be referred to as the preferential attachment function

\end{definition}

\hypertarget{expected-degree-dristribution}{%
\subsubsection{Expected Degree
Dristribution}\label{expected-degree-dristribution}}

In {[}GPA REF{]} the expected degree distribution for \(m=1\) was
calculated in terms of the preferential attachment function does not
have a general explicit form. It is defined as follows, let
\(\lambda^*\) be the solution, if it exists, to:

\[
1=\sum_{n=1}^\infty \prod_{i=1}^{n-1}\frac{g(i)}{g(i)+\lambda}
\] then the expected degree distribution resulting from the GPA model
has pmf:

\[
f(k) = \frac{\lambda^*}{g(k) + \lambda^*}\prod_{i=0}^{k-1}\frac{g(i)}{g(i)+\lambda^*}
\]

\hypertarget{barabuxe1si-albert-ba}{%
\subsection{Barabási-Albert (BA)}\label{barabuxe1si-albert-ba}}

The first special case is the Barabási-Albert model, which is equivalent
to setting the preferential attachment function \(g\) to be the identity
function \(g(k)=k\)

This model defined in Barabási and Albert (1999) and also very closely
related to the Yule-Simon process from {[}YS REF{]} changes the
attachment mechanism from being purely uniform on the vertices already
in the network to being random with a probability proportional to the
degrees of the vertices in the network.

\begin{definition}[Barabási-Albert
Model]\protect\hypertarget{def-ba}{}\label{def-ba}

Starting with a graph \(G_1 = (V_1, E_1)\) where
\(V_1 = \{1,\ldots,m_0\}\) and \(E_1 = \{\{v\}:v\in V_1\}\) i.e a graph
with \(m_0\) vertices with one self-loop each. At each time step \(t>1\)
the graph denoted by \(G_t = (V_1, E_1)\) is generated by repeating the
following:

\begin{enumerate}
\def\labelenumi{\arabic{enumi}.}
\tightlist
\item
  \textbf{Growth:} Add a new vertex to the vertex set i.e. \[
  V_t = V_{t-1} \cup \{t\}
  \]
\item
  \textbf{Preferential Attachment:} Add \(m\le m_0\) edges between the
  new vertex and those already in the graph with probability
  proportional to each vertices degree i.e:
\end{enumerate}

\[
E_t= E_{t-1} \cup \tilde E
\] where \(\tilde E = \{\tilde e_1,\ldots, \tilde e_m\}\) and
\(\tilde e_i = \{t,\tilde v_i\}\) for
\(\tilde v_i \sim \text{Cat}(V_{t-1}, P)\)

\begin{align*}
P &= \left\{\frac{d(v)}{\sum_{w\in V_{t-1}} d(w)} : v \in V_{t-1}\right\}
\end{align*}

\end{definition}

\hypertarget{expected-degree-distriubtion}{%
\subsubsection{Expected Degree
Distriubtion}\label{expected-degree-distriubtion}}

In the same paper (Barabási and Albert (1999)) it was shown that for
large values of \(t\) the expected degree distribution for this model is
approximately: \[
f(k) = \frac{2m^2t}{m_0+t}k^{-3} \approx 2m^2k^{-3},\qquad k\ge m
\]

This is clearly a regularly varying function and therefore in in the
Fréchet domain of attraction \(\mathcal D(\Phi_2)\).

\hypertarget{uniform-attachment-ua}{%
\subsection{Uniform Attachment (UA)}\label{uniform-attachment-ua}}

The final special case presented here is obtained from setting the
preferential attachment function \(g\) to be some constant value.

\begin{definition}[Uniform Attachment
Model]\protect\hypertarget{def-ua}{}\label{def-ua}

Start with a graph \(G_1 = (V_1, E_1) = (\{1,\ldots,m_0\}, \emptyset)\),
at each time step \(t>1\) the graph is denoted by \(G_t=(V_t, E_t)\) and
generated by repeating the following two steps:

\begin{enumerate}
\def\labelenumi{\arabic{enumi}.}
\tightlist
\item
  \textbf{Growth:} Add a new vertex to the vertex set i.e.~ \[
  V_t=V_{t-1}\cup\{t\}
  \]
\item
  \textbf{Attachment:} Add \(m\le m_0\) random edges between the new
  vertex and those already in the graph i.e.~ \[
  E_t = E_{t-1} \cup \tilde E
  \] where \(\tilde E = \{\tilde e_1,\ldots, \tilde e_m\}\) and
  \(\tilde e_i = \{t,\tilde v_i\}\) and \(\tilde v_i \sim U(V_{t-1})\).
\end{enumerate}

\end{definition}

\hypertarget{expected-degree-distribution}{%
\subsubsection{Expected Degree
Distribution}\label{expected-degree-distribution}}

As showing in {[}REF{]} the expected degree distribution of this model
for large values of \(t\) is: \[
f(k) = \frac{e}{m}\exp\left(-\frac{k}{m}\right),\qquad k \ge m
\] Since this distribution has exponential form, it is in the Gumbel
domain of attraction.

\hypertarget{sec-meth}{%
\chapter{Methods}\label{sec-meth}}

As mentioned in Section~\ref{sec-mod}, the method used here to model the
extreme values of the data will be a spliced threshold mixture.
Specifically, it will be a spliced threshold mixture of a power law and
a discretisation of the generalised pareto distribution similar to what
is defined in {[}ROHRBECK{]}.

\begin{definition}[Intergral Generalised Pareto Distribution
(IGP)]\protect\hypertarget{def-igp}{}\label{def-igp}

Consider a random variable \(X\) with cdf \(F\), and consider the random
variable \(Y=\lfloor X \rfloor\). From Definition~\ref{def-gp},
\(X|X>u \sim GP(\sigma, \xi)\) for some sufficiently large
\(u\in \mathbb R^+\) and it can be obtained that the distribution of
\(Y|Y>u\) has distribution defined below:

\[
\Pr(Y=y>Y>u) = \left(1+\frac{\xi(y+1-\lceil u\rceil)}{\sigma_0+\xi\lceil u\rceil}\right)_+^{-1/\xi}-\left(1+\frac{\xi(y-\lceil u\rceil)}{\sigma_0+\xi\lceil u\rceil}\right)_+^{-1/\xi}
\]

For \(y=\lceil u\rceil,\lceil u\rceil+1, \ldots\) and
\(\xi \in \mathbb R\) and \(u, \sigma_0 \in \mathbb R^+.\)

\end{definition}

Since the some degree distributions of real networks seen in {[}FIG{]}
seem to be approximately linear for the bulk of the data and then begin
to change, the spliced threshold mixture that will be used consists of a
truncated discrete power law for the bulk of the data and a GP above a
threshold.

\begin{definition}[Power-Law IGP
Distribution]\protect\hypertarget{def-pligp}{}\label{def-pligp}

\[
f(y) = \begin{cases}
(1-\phi)\frac{y^{-(\alpha+1})}{\sum_{k=1}^v}, & y=1,2,\ldots, v\\
\phi\left[\left(1+\frac{\xi(y+1-v)}{\sigma_0+\xi v}\right)_+^{-1/\xi}-\left(1+\frac{\xi(y-v)}{\sigma_0+\xi v}\right)_+^{-1/\xi}\right],&y=v+1, v+2,\ldots
\end{cases}
\]

\end{definition}

\hypertarget{fitting-model-to-real-data}{%
\section{Fitting model to real data}\label{fitting-model-to-real-data}}

\hypertarget{gpa-analyses}{%
\section{GPA analyses}\label{gpa-analyses}}

\hypertarget{conclusion-and-a-conjecture}{%
\section{Conclusion and a
Conjecture}\label{conclusion-and-a-conjecture}}

\hypertarget{next-steps}{%
\chapter{Next Steps}\label{next-steps}}

\newpage{}

\hypertarget{references}{%
\chapter*{References}\label{references}}
\addcontentsline{toc}{chapter}{References}

\hypertarget{refs}{}
\begin{CSLReferences}{1}{0}
\leavevmode\vadjust pre{\hypertarget{ref-agn22}{}}%
Ahmad, Touqeer, Carlo Gaetan, and Philippe Naveau. 2022. {``Modelling of
Discrete Extremes Through Extended Versions of Discrete Generalized
Pareto Distribution.''} \emph{ArXiv e-Prints}.
\url{https://arxiv.org/abs/2210.15253}.

\leavevmode\vadjust pre{\hypertarget{ref-Barabasi99}{}}%
Barabási, Albert-László, and Réka Albert. 1999. {``Emergence of Scaling
in Random Networks.''} \emph{Science} 286 (5439): 509--12.
\url{https://doi.org/10.1126/science.286.5439.509}.

\leavevmode\vadjust pre{\hypertarget{ref-fmh09}{}}%
Fraga Alves, Maria, Laurens Haan, and Cláudia Neves. 2009. {``A Test
Procedure for Detecting Super-Heavy Tails.''} \emph{Journal of
Statistical Planning and Inference} 139 (February).
\url{https://doi.org/10.1016/j.jspi.2008.04.026}.

\leavevmode\vadjust pre{\hypertarget{ref-hds24}{}}%
Hitz, Adrien S., Richard A. Davis, and Gennady Samorodnitsky. 2024.
{``Discrete Extremes.''} \emph{Journal of Data Science}, 1--13.
\url{https://doi.org/10.6339/24-JDS1120}.

\leavevmode\vadjust pre{\hypertarget{ref-shimura12}{}}%
Shimura, Takaaki. 2012. {``Discretization of Distributions in the
Maximum Domain of Attraction.''} \emph{Extremes} 15: 299--317.
\url{https://doi.org/10.1007/s10687-011-0137-7}.

\end{CSLReferences}



\end{document}
