% Options for packages loaded elsewhere
\PassOptionsToPackage{unicode}{hyperref}
\PassOptionsToPackage{hyphens}{url}
\PassOptionsToPackage{dvipsnames,svgnames,x11names}{xcolor}
%
\documentclass[
  10pt,
  a4paper,
]{scrreprt}

\usepackage{amsmath,amssymb}
\usepackage{iftex}
\ifPDFTeX
  \usepackage[T1]{fontenc}
  \usepackage[utf8]{inputenc}
  \usepackage{textcomp} % provide euro and other symbols
\else % if luatex or xetex
  \usepackage{unicode-math}
  \defaultfontfeatures{Scale=MatchLowercase}
  \defaultfontfeatures[\rmfamily]{Ligatures=TeX,Scale=1}
\fi
\usepackage{lmodern}
\ifPDFTeX\else  
    % xetex/luatex font selection
\fi
% Use upquote if available, for straight quotes in verbatim environments
\IfFileExists{upquote.sty}{\usepackage{upquote}}{}
\IfFileExists{microtype.sty}{% use microtype if available
  \usepackage[]{microtype}
  \UseMicrotypeSet[protrusion]{basicmath} % disable protrusion for tt fonts
}{}
\makeatletter
\@ifundefined{KOMAClassName}{% if non-KOMA class
  \IfFileExists{parskip.sty}{%
    \usepackage{parskip}
  }{% else
    \setlength{\parindent}{0pt}
    \setlength{\parskip}{6pt plus 2pt minus 1pt}}
}{% if KOMA class
  \KOMAoptions{parskip=half}}
\makeatother
\usepackage{xcolor}
\usepackage[inner=2cm,outer=2cm,top=2cm,bottom=2cm,headsep=22pt,headheight=11pt,footskip=33pt,ignorehead,ignorefoot,heightrounded]{geometry}
\setlength{\emergencystretch}{3em} % prevent overfull lines
\setcounter{secnumdepth}{1}
% Make \paragraph and \subparagraph free-standing
\ifx\paragraph\undefined\else
  \let\oldparagraph\paragraph
  \renewcommand{\paragraph}[1]{\oldparagraph{#1}\mbox{}}
\fi
\ifx\subparagraph\undefined\else
  \let\oldsubparagraph\subparagraph
  \renewcommand{\subparagraph}[1]{\oldsubparagraph{#1}\mbox{}}
\fi


\providecommand{\tightlist}{%
  \setlength{\itemsep}{0pt}\setlength{\parskip}{0pt}}\usepackage{longtable,booktabs,array}
\usepackage{calc} % for calculating minipage widths
% Correct order of tables after \paragraph or \subparagraph
\usepackage{etoolbox}
\makeatletter
\patchcmd\longtable{\par}{\if@noskipsec\mbox{}\fi\par}{}{}
\makeatother
% Allow footnotes in longtable head/foot
\IfFileExists{footnotehyper.sty}{\usepackage{footnotehyper}}{\usepackage{footnote}}
\makesavenoteenv{longtable}
\usepackage{graphicx}
\makeatletter
\def\maxwidth{\ifdim\Gin@nat@width>\linewidth\linewidth\else\Gin@nat@width\fi}
\def\maxheight{\ifdim\Gin@nat@height>\textheight\textheight\else\Gin@nat@height\fi}
\makeatother
% Scale images if necessary, so that they will not overflow the page
% margins by default, and it is still possible to overwrite the defaults
% using explicit options in \includegraphics[width, height, ...]{}
\setkeys{Gin}{width=\maxwidth,height=\maxheight,keepaspectratio}
% Set default figure placement to htbp
\makeatletter
\def\fps@figure{htbp}
\makeatother
\newlength{\cslhangindent}
\setlength{\cslhangindent}{1.5em}
\newlength{\csllabelwidth}
\setlength{\csllabelwidth}{3em}
\newlength{\cslentryspacingunit} % times entry-spacing
\setlength{\cslentryspacingunit}{\parskip}
\newenvironment{CSLReferences}[2] % #1 hanging-ident, #2 entry spacing
 {% don't indent paragraphs
  \setlength{\parindent}{0pt}
  % turn on hanging indent if param 1 is 1
  \ifodd #1
  \let\oldpar\par
  \def\par{\hangindent=\cslhangindent\oldpar}
  \fi
  % set entry spacing
  \setlength{\parskip}{#2\cslentryspacingunit}
 }%
 {}
\usepackage{calc}
\newcommand{\CSLBlock}[1]{#1\hfill\break}
\newcommand{\CSLLeftMargin}[1]{\parbox[t]{\csllabelwidth}{#1}}
\newcommand{\CSLRightInline}[1]{\parbox[t]{\linewidth - \csllabelwidth}{#1}\break}
\newcommand{\CSLIndent}[1]{\hspace{\cslhangindent}#1}

\addtokomafont{disposition}{\rmfamily}
\usepackage{ragged2e}
\usepackage{blindtext}\usepackage{amsthm}
\usepackage{amsthm}
\usepackage{hyperref}
\newtheorem{thm}{Theorem}[subsection]
\renewcommand{\thethm}{\arabic{subsection}.\arabic{thm}}

\makeatletter
\makeatother
\makeatletter
\makeatother
\makeatletter
\@ifpackageloaded{caption}{}{\usepackage{caption}}
\AtBeginDocument{%
\ifdefined\contentsname
  \renewcommand*\contentsname{Table of contents}
\else
  \newcommand\contentsname{Table of contents}
\fi
\ifdefined\listfigurename
  \renewcommand*\listfigurename{List of Figures}
\else
  \newcommand\listfigurename{List of Figures}
\fi
\ifdefined\listtablename
  \renewcommand*\listtablename{List of Tables}
\else
  \newcommand\listtablename{List of Tables}
\fi
\ifdefined\figurename
  \renewcommand*\figurename{Figure}
\else
  \newcommand\figurename{Figure}
\fi
\ifdefined\tablename
  \renewcommand*\tablename{Table}
\else
  \newcommand\tablename{Table}
\fi
}
\@ifpackageloaded{float}{}{\usepackage{float}}
\floatstyle{ruled}
\@ifundefined{c@chapter}{\newfloat{codelisting}{h}{lop}}{\newfloat{codelisting}{h}{lop}[chapter]}
\floatname{codelisting}{Listing}
\newcommand*\listoflistings{\listof{codelisting}{List of Listings}}
\usepackage{amsthm}
\theoremstyle{plain}
\newtheorem{theorem}{Theorem}[section]
\theoremstyle{definition}
\newtheorem{definition}{Definition}[section]
\theoremstyle{remark}
\AtBeginDocument{\renewcommand*{\proofname}{Proof}}
\newtheorem*{remark}{Remark}
\newtheorem*{solution}{Solution}
\makeatother
\makeatletter
\@ifpackageloaded{caption}{}{\usepackage{caption}}
\@ifpackageloaded{subcaption}{}{\usepackage{subcaption}}
\makeatother
\makeatletter
\@ifpackageloaded{tcolorbox}{}{\usepackage[skins,breakable]{tcolorbox}}
\makeatother
\makeatletter
\@ifundefined{shadecolor}{\definecolor{shadecolor}{rgb}{.97, .97, .97}}
\makeatother
\makeatletter
\makeatother
\makeatletter
\makeatother
\ifLuaTeX
  \usepackage{selnolig}  % disable illegal ligatures
\fi
\IfFileExists{bookmark.sty}{\usepackage{bookmark}}{\usepackage{hyperref}}
\IfFileExists{xurl.sty}{\usepackage{xurl}}{} % add URL line breaks if available
\urlstyle{same} % disable monospaced font for URLs
\hypersetup{
  pdftitle={Annual Progress Review},
  pdfauthor={Thomas William Boughen},
  colorlinks=true,
  linkcolor={blue},
  filecolor={Maroon},
  citecolor={Blue},
  urlcolor={Blue},
  pdfcreator={LaTeX via pandoc}}

\title{Annual Progress Review}
\author{Thomas William Boughen}
\date{}

\begin{document}
\cleardoublepage
\thispagestyle{empty}
{\centering
\hbox{}\vskip 0cm plus 1fill
{\Huge\bfseries Annual Progress Review \par}
\vspace{12ex}
{\Large\bfseries Thomas William Boughen \par}
\vspace{3ex}
\vskip 0cm plus 2fill
%{\bfseries\large Doctor of Philosophy \par}
\vspace{3ex}
{\bfseries\large  \par}
\vspace{12ex}
{\includegraphics[width=0.1\linewidth]{"imgs/University_of_Newcastle_Coat_of_Arms.png"}\par}
%
%
{\bfseries\large Newcastle University \par}
\vspace{3ex}
%
{\bfseries\large School of Mathematics, Statistics and Physics \par}
%
\vspace{12ex}
%{\small Submitted in total fulfilment of the requirements
%of the degree of Doctor of Philosophy \par}
%}
\footnote{An html version of this report can be found at \url{twboughen.github.io/phd/APR/report}}
\justifying
\noindent\ifdefined\Shaded\renewenvironment{Shaded}{\begin{tcolorbox}[enhanced, sharp corners, breakable, boxrule=0pt, frame hidden, interior hidden, borderline west={3pt}{0pt}{shadecolor}]}{\end{tcolorbox}}\fi

\hypertarget{sec-int}{%
\chapter{Intuition}\label{sec-int}}

\hypertarget{sec-ext}{%
\chapter{Extremes}\label{sec-ext}}

Since the aim is to gain understanding about the behaviour of the degree
distribution of networks at the right tail, it seems natural to look to
using methods from extreme value theory. However, networks by their
nature are discrete and so it may not be best to be using methods that
are usually used in relation to continuous random variables. For this
reason, this section starts with a review of what theory exists for
modelling the extreme values of continuous random variables before
moving to details what can be used when instead considering discrete
random variables as is the case for the degree distributions of random
networks.

\hypertarget{sec-ce}{%
\section{Continuous Extremes}\label{sec-ce}}

Studying the properties of the right tail of the distribution of a
continuous random variable, means that the focus is on the largest
values that the random variable can take. So, a natural place to start
is to consider the distribution of the block maxima of such a random
variable. That is, for a set of iid random variables
\(\{X_1,\ldots,X_n\}\) with common cumulative density function (cdf)
\(F\) what is the distribution of \(M_n = \max\{X_1,\ldots,X_n\}\)? This
question is answered by the Fisher--Tippett--Gnedenko theorem
\href{or\%20more\%20simply\%20the\%20extreme\%20value\%20theorem}{REF}.

\begin{theorem}[Extreme Value
Theorem]\protect\hypertarget{thm-evt}{}\label{thm-evt}

Let \(X_1,\ldots,X_n\) be a sample of iid random variables with common
cdf \(F\) with block maxima \(M_n = \max\{X_1,\ldots,X_n\}\) and suppose
that there exists \(a_n>0, b_n\in\mathbb R\) such that
\(\lim_{n\rightarrow\infty}\Pr(\frac{1}{a_n}[M_n-b_n]) = G(x)\), then
\(F\) is said to be in the domain of attraction of \(G\), denoted
\(F\in\mathcal D(G)\) ,and \(G\) is of one of three types:

\begin{itemize}
\tightlist
\item
  Gumbel: \(\Lambda(x) = \exp\{-\exp(-x)\},\quad x \in \mathbb R\)
\item
  Fréchet:
  \(\Phi_\alpha(x) = \exp\{-x^{-\alpha}\},\quad x\ge 0,\alpha>0\)
\item
  Weibull: \(\Psi_\alpha(x) = \exp\{-x^{-a}\},\quad x<0,\alpha>0\)
\end{itemize}

\end{theorem}

While this is a useful result, it may prove difficult to find the
sequences \(a_n,b_n\) in practice, so a simpler method to establish what
domain of attraction a distribution belongs to would be nice. Luckily,
this can be done through the concept of regular variation and is what
will be used to define the domains of attraction and tail-heaviness
through the rest of this report.

\begin{definition}[Domains of
Attraction]\protect\hypertarget{def-doa}{}\label{def-doa}

The distribution \(F\) belongs to the Fréchet domain of attraction
\(\mathcal D(\Phi_\alpha)\) if and only if its complement (the survival
function) \(\bar F\) is regularly varying with index \(-\alpha\) i.e.:
\[
\bar F(x) = x^{-\alpha}L(x),\qquad \text{for } L \text{ slowly varying}
\] A similar condition applies to the Weibull domain of attraction
\(\mathcal D(\Psi_\alpha)\) in that a distribution \(F\) belongs to the
Weibull domain of attraction if and only if: \[
\bar F(x_F-x^{-1}) = x^{-\alpha}L(x),\qquad \text{for } L \text{ slowly varying}
\] where \(x_F\) is the finite right endpoint of the support of \(F\).

The condition for the Gumbel domain of attraction is not as simple, a
distribution \(F\) belongs to the Gumbel domain if and only if there
exists a positive function \(a: \mathbb R \rightarrow \mathbb R^+\) and
a \(t\in \mathbb R\) such that: \[
\lim_{x\rightarrow x_F} \frac{\bar F(x+ta(x))}{\bar F(x)} = e^{-t}
\]

\end{definition}

Throughout this report the term ``heavy tailed'' distribution will be
used to describe any distribution in the Fréchet domain of attraction,
although some of the literature refers to ``heavy tailed'' distributions
as being the distributions that decay slower than the exponential.

At this point it will also be useful to introduce the concept of
distributions that have super-heavy tails.

\begin{definition}[Super Heavy
Tails]\protect\hypertarget{def-sht}{}\label{def-sht}

{[}FIND SUPER HEAVY TAILS DEFINITION{]}

Additionally, if the survival function \(\bar F\) is slowly varying
itself then \(F\) has super heavy tails i.e. \[
\lim_{x\rightarrow\infty}\frac{\bar F(tx)}{\bar{F}(x)} = 1, \forall t\in\mathbb R^+ \implies \text{super heavy tails} 
\]

\end{definition}

It is possible to gather the main three types of extremal distributions
into what is called the Generalised Extreme Value (GEV) distribution
{[}REF{]}.

\begin{definition}[Generalised Extreme Value
Distribution]\protect\hypertarget{def-gev}{}\label{def-gev}

Denoted by \(\text{GEV}(\mu,\sigma,\xi)\) the distribution is
characterised by three parameters \(\mu \in \mathbb R\) the location,
\(\sigma\in \mathbb R^+\) the scale, and the shape \(\xi\in \mathbb R\).
It has support on \(\{x\in \mathbb R:1+\xi(x-\mu)/\sigma > 0\}\) and has
cdf given by:

\[
G(x) = \begin{cases}\exp\left\{-\left(1+\frac{\xi(x-\mu)}{\sigma}\right)^{-1/\xi}\right\},&\xi\ne0\\
\exp\{-\exp(-\frac{x-\mu}{\sigma})\},&\xi=0
\end{cases}
\]

\end{definition}

The three types of extremal distribution are obtained from changing the
shape parameter \(\xi\), which corresponds to \(1/\alpha\) in the
definition of the domains of attraction. This change is generally to
made so that increasing \(\xi\) corresponds to increasing how heavy the
tails of the distribution are. So, \(\xi<0\) corresponds to the Weibull,
\(\xi>0\) the Fréchet, and \(\xi=0\) the Gumbel.

While this is useful for modelling the distribution of block maxima of
iid random variables, as seen in Section~\ref{sec-int}, the data in
question appears to follow power law like behaviour for the bulk of the
data and then changes to various different shapes above a certain
threshold. For this reason, it is perhaps more appropriate to consider
the distribution of threshold exceedances.

\begin{definition}[Generalised Pareto
Distribution]\protect\hypertarget{def-gp}{}\label{def-gp}

The Generalised Pareto (GP) distribution can be obtained by using the
GEV distribution and conditional probability such that for large enough
threshold the GP distribution approximately describes the conditional
distribution of threshold exceedances. More precisely, for large enough
threshold \(u\) and the change of variable to \(Y=X-u\): \[
\Pr(Y\le y | Y>0) = H(y) = \begin{cases}
1-\left(1+\frac{\xi y}{\sigma}\right)^{-1/\xi},&y>0,\xi\ne 0 \\
1-\exp\left(-\frac{y}{\sigma}\right),&y>0,\xi = 0
\end{cases}
\]

\end{definition}

Since this distribution was obtained using a
\(\text{GEV}(\mu,\sigma^*,\xi)\) the shape parameter \(\xi\) is
identical in both distributions and the shape parameter \(\sigma\) is
defined such that \(\sigma = \sigma^* + \xi(u-\mu)\).

The vast majority of this theory is appropriate only for continuous
data, and since the data being focused on is discrete, some results for
discrete extremes should be introduced.

\hypertarget{discrete-extremes}{%
\section{Discrete Extremes}\label{discrete-extremes}}

Moving to modelling extremes in a discrete has the potential to cause
some issues when describing how heavy the tails of a distribution are
and what domain of attraction belongs to. For example, the exponential
distribution belongs to the Gumbel domain \(\mathcal D(\Lambda)\) but
its discrete counterpart (the geometric distribution) does not belong to
the Gumbel domain. So, care needs to be taken when attempting to
discretise the results from Section~\ref{sec-ce}.

Shimura (2012) provides conditions for a discrete distribution to belong
to the domain of attraction. In particular the following theorem which
corresponds to Theorem 1 in Shimura (2012).

\begin{theorem}[Discrete Domains of
Attraction]\protect\hypertarget{thm-shimura1}{}\label{thm-shimura1}

~

\begin{enumerate}
\def\labelenumi{(\alph{enumi})}
\tightlist
\item
  Every discretisation of distribution in \(\mathcal D(\Phi_\alpha)\)
  remains in \(\mathcal D(\Phi_\alpha)\).
\item
  The discretisation of a distribution remains in
  \(\mathcal D(\Lambda)\) if and only if the original is in
  \(\mathcal D(\Lambda)\cap \mathcal L\).
\end{enumerate}

Where \(\mathcal L\) is the set of long-tailed distributions that have
the property: \[
\lim_{x\rightarrow \infty}\frac{\overline F(x+1)}{\overline F(x)} = 1   
\]

\end{theorem}

In addition to this theorem, Shimura also introduces a quantity that
will become useful when deciding what domain of attraction a discrete
distribution belongs to. In this report we will simply refer to it as
the Omega function and is defined below:

\begin{definition}[Omega
Function]\protect\hypertarget{def-omega}{}\label{def-omega}

For a distribution \(F\) with survival function \(\overline F\) and some
\(n\in\mathbb Z^+\) let:

\[
\Omega(F,n) = \left(\log\frac{\overline F (n+1)}{\overline F (n+2)}\right)^{-1} - \left(\log\frac{\overline F (n)}{\overline F (n+1)}\right)^{-1}
\]

\end{definition}

This quantity will play an important role in Section~\ref{sec-meth} when
determining what kinds of degree distributions different network
generative models are expected to lead to.

Applying ideas from Section~\ref{sec-ce} to modelling discrete random
variables has been approached from many different directions. What
follows is a overview of some of the approaches that have been taken but
will see use in this report.

One of the main issues when it comes to applying results from
Section~\ref{sec-ce} to discrete data, is the discretisation of the GP
distribution and making sure that it still maintains most of the same
properties.

{[}AHMAD{]} introduces several ways to discretise the GP distribution
including mixing the geometric distribution with a Gamma distribution,
they also introduce an extended GP distribution (and its discretisation)
which is obtained from a probability integral transform (PIT) of a
distribution with support on \((0,1)\). This is a modification of how
the regular GP is obtained, which is from a PIT uniform distribution on
(0,1).

Also considering mixtures, {[}VALIQUETTE{]} investigated the tail
properties of Poisson mixtures and found that changing the mixing
distribution changes the tail heaviness of the mixed distribution. They
divided the domain of attraction of the mixing distribution into various
subsets that lead to different tail behaviours, through the limit of the
ratio of consecutive values of the survival function. These limits do
help describe the tail behaviour of the resulting mixture but do not
quite correspond to different domains of attraction.

The approach used in Section~\ref{sec-meth} will be very similar to the
one taken in {[}ROHRBECK{]} with a minor change to the derivation, this
will be explained in detail in Section~\ref{sec-meth}.

\hypertarget{networks}{%
\chapter{Networks}\label{networks}}

Networks are the structures that will be the source of data that the
results from Section~\ref{sec-ext} will be used to analyse. Networks
appear across a wide range of fields when attempting to represent
complex systems and the relationships between the components within,
showing up in anything from micro-biology (e.g.~protein interactions in
cells) to sociology (e.g.~the social network of Harvard graduates).

This makes networks a valuable source of data and understanding the
mechanics of the network generation process can provide insights to the
components themselves and into the networks future.

\hypertarget{mathematical-definitions}{%
\section{Mathematical Definitions}\label{mathematical-definitions}}

Networks on the face of it are fairly simple objects, nothing more than
a collection of objects with connections between each other. Here,
graphs constructed from vertices and edges will be used as an analogue
for these networks.

\begin{definition}[Graph/Network]\protect\hypertarget{def-net}{}\label{def-net}

A graph \(G = (V,E)\) is constructed from a vertex set
\(V\in\mathbb Z^+\) and an edge set \(E\). The edge set can take on one
of two forms depending on if the graph is directed or un-directed. If
the graph is directed then \(E\subseteq V^2\) i.e the edge set is
contained within the set of ordered pairs of vertices, whereas if the
graph is \textbf{un-directed} then \(E\subseteq [V]^2\) i.e.~the edge
set is contained within the set of un-ordered pairs of vertices. The
focus from now on will be on un-directed networks and graphs.

\end{definition}

Throughout this section and the next the concept of a vertices
``degree'' will come up, and in fact the main focus of
Section~\ref{sec-meth} is the degree distribution of networks.

\begin{definition}[Degree]\protect\hypertarget{def-deg}{}\label{def-deg}

For an un-directed graph a vertex's degree denoted \(d(v)\) or \(k_v\)
for \(v\in V\) is the number of edges that are connected to vertex
\(v\): \[
d(v) = |\{e\in E : v \in e\}|
\] Directed graphs have something analogous, called the in-degree
\(d_{in}\), out-degree \(d_{out}\) and total degree \(d_{tot}\). The
in-degree of a vertex \(v\) is the number edges with endpoint at \(v\),
whereas the out-degree is the number of edges with start point at \(v\)
and the total degree is the sum of these i.e.:

\begin{align*}
d_{in}(v)&= |\{(w_1,w_2)\in E: w_2=v \}|\\
d_{out}(v) &= |\{(w_1,w_2)\in E: w_1=v \}|\\
d_{tot}(v) &= d_{in}(v) + d_{out}(v)
\end{align*}

\end{definition}

There are many reasons to analyse network like data, one of which is to
gain an insight into the mechanics that governed the growth of the
network. The next sub-section is focused on presenting several network
generative models increasing in generality.

\hypertarget{network-generative-models}{%
\section{Network Generative Models}\label{network-generative-models}}

The models in this section are nested within one another, they all
follow the same regiment that at each time-step a vertex is added to the
network along with one or more edges that connect the new vertex to
those already in the network.

\hypertarget{uniform-attachment-ua}{%
\subsection{Uniform Attachment (UA)}\label{uniform-attachment-ua}}

The first model dubbed the ``uniform attachment model'' is a simple
model first presented in {[}UA REF{]} and is defined as follows:

\begin{definition}[Uniform Attachment
Model]\protect\hypertarget{def-ua}{}\label{def-ua}

Start with a graph \(G_1 = (V_1, E_1) = (\{1,\ldots,m_0\}, \emptyset)\),
at each time step \(t>1\) the graph is denoted by \(G_t=(V_t, E_t)\) and
generated by repeating the following two steps:

\begin{enumerate}
\def\labelenumi{\arabic{enumi}.}
\tightlist
\item
  \textbf{Growth:} Add a new vertex to the vertex set i.e.~ \[
  V_t=V_{t-1}\cup\{t\}
  \]
\item
  \textbf{Attachment:} Add \(m\le m_0\) random edges between the new
  vertex and those already in the graph i.e.~ \[
  E_t = E_{t-1} \cup \tilde E
  \] where \(\tilde E = \{\tilde e_1,\ldots, \tilde e_m\}\) and
  \(\tilde e_i = \{t,\tilde v_i\}\) and \(\tilde v_i \sim U(V_{t-1})\).
\end{enumerate}

\end{definition}

\hypertarget{expected-degree-distribution}{%
\subsubsection{Expected Degree
Distribution}\label{expected-degree-distribution}}

As showing in {[}REF{]} the expected degree distribution of this model
for large values of \(t\) is: \[
f(k) = \frac{e}{m}\exp\left(-\frac{k}{m}\right),\qquad k \ge m
\] Since this distribution has exponential form, it is in the Gumbel
domain of attraction.

\hypertarget{barabuxe1si-albert-ba}{%
\subsection{Barabási-Albert (BA)}\label{barabuxe1si-albert-ba}}

This model defined in Barabási and Albert (1999) and also very closely
related to the Yule-Simon process from {[}YS REF{]} changes the
attachment mechanism from being purely uniform on the vertices already
in the network to being random with a probability proportional to the
degrees of the vertices in the network.

\begin{definition}[Barabási-Albert
Model]\protect\hypertarget{def-ba}{}\label{def-ba}

Starting with a graph \(G_1 = (V_1, E_1)\) where
\(V_1 = \{1,\ldots,m_0\}\) and \(E_1 = \{\{v\}:v\in V_1\}\) i.e a graph
with \(m_0\) vertices with one self-loop each. At each time step \(t>1\)
the graph denoted by \(G_t = (V_1, E_1)\) is generated by repeating the
following:

\begin{enumerate}
\def\labelenumi{\arabic{enumi}.}
\tightlist
\item
  \textbf{Growth:} Add a new vertex to the vertex set i.e. \[
  V_t = V_{t-1} \cup \{t\}
  \]
\item
  \textbf{Preferential Attachment:} Add \(m\le m_0\) edges between the
  new vertex and those already in the graph with probability
  proportional to each vertices degree i.e:
\end{enumerate}

\[
E_t= E_{t-1} \cup \tilde E
\] where \(\tilde E = \{\tilde e_1,\ldots, \tilde e_m\}\) and
\(\tilde e_i = \{t,\tilde v_i\}\) for
\(\tilde v_i \sim \text{Cat}(V_{t-1}, P)\)

\begin{align*}
P &= \left\{\frac{d(v)}{\sum_{w\in V_{t-1}} d(w)} : v \in V_{t-1}\right\}
\end{align*}

\end{definition}

\hypertarget{expected-degree-distriubtion}{%
\subsubsection{Expected Degree
Distriubtion}\label{expected-degree-distriubtion}}

In the same paper (Barabási and Albert (1999)) it was shown that for
large values of \(t\) the expected degree distribution for this model is
approximately: \[
f(k) = \frac{2m^2t}{m_0+t}k^{-3} \approx 2m^2k^{-3},\qquad k\ge m
\]

This is clearly a regularly varying function and therefore in in the
Fréchet domain of attraction \(\mathcal D(\Phi_2)\).

\hypertarget{general-preferential-attachment-gpa}{%
\subsection{General Preferential Attachment
(GPA)}\label{general-preferential-attachment-gpa}}

This model generalises the BA model by instead of attaching the new
edges with a probability proportional to vertices' degrees, the new
edges are attached with probability proportional to some function of the
vertices' degrees.

\begin{definition}[General Preferential Attachment
Model]\protect\hypertarget{def-gpa}{}\label{def-gpa}

Starting with a graph
\(G_1 = (V_1, E_1) = (\{1,\ldots,m_0\}, \emptyset)\), at each following
time step \(t>1\) the graph is denoted by \(G_t = (V_t, E_t)\) and is
generated by repeating:

\begin{enumerate}
\def\labelenumi{\arabic{enumi}.}
\tightlist
\item
  \textbf{Growth:} Add a new vertex to the vertex set i.e. \[
  V_t = V_{t-1} \cup \{t\}
  \]
\item
  \textbf{Preferential Attachment:} Add \(m\le m_0\) edges connecting
  the new vertex to those already in the graph selected at random
  proportional to a function of their degree(minus one)\footnote{The
    probabilities are proportional to the degree minus one to align with
    the results from {[}GPA REF{]}} i.e.: \[
  E_t  = E_{t-1} \cup \tilde E
  \] where \(\tilde E = \{\tilde e_1,\ldots, \tilde e_m\}\) and
  \(\tilde e_i = \{t,\tilde v_i\}\) for
  \(\tilde v_i \sim \text{Cat}(V_{t-1}, P)\)
\end{enumerate}

\begin{align*}
P &= \left\{\frac{g(k_v-1)}{\sum_{w\in V_{t-1}} g(k_w-1)} : v \in V_{t-1}\right\}
\end{align*}

for some function \(g: \mathbb Z \mapsto \mathbb R^+\setminus\{0\}\),
which will be referred to as the preferential attachment function

\end{definition}

\hypertarget{expected-degree-dristribution}{%
\subsubsection{Expected Degree
Dristribution}\label{expected-degree-dristribution}}

In {[}GPA REF{]} the expected degree distribution for \(m=1\) was
calculated in terms of the preferential attachment function does not
have a general explicit form. It is defined as follows, let
\(\lambda^*\) be the solution, if it exists, to:

\[
1=\sum_{n=1}^\infty \prod_{i=1}^{n-1}\frac{g(i)}{g(i)+\lambda}
\] then the expected degree distribution resulting from the GPA model
has pmf:

\[
f(k) = \frac{\lambda^*}{g(k) + \lambda^*}\prod_{i=0}^{k-1}\frac{g(i)}{g(i)+\lambda^*}
\] It is hard to see how this may behave in the tails for different
preferential attachment functions, this will be one of the main focuses
of the next section.

\hypertarget{sec-meth}{%
\chapter{Methods}\label{sec-meth}}

\hypertarget{integral-generalised-pareto-distribution}{%
\section{Integral Generalised Pareto
Distribution}\label{integral-generalised-pareto-distribution}}

\hypertarget{mixture-model}{%
\section{Mixture Model}\label{mixture-model}}

\hypertarget{threshold-selection}{%
\subsection{Threshold Selection}\label{threshold-selection}}

\hypertarget{fitting-model-to-simulated-data}{%
\section{Fitting model to simulated
data}\label{fitting-model-to-simulated-data}}

\hypertarget{ua-model}{%
\subsection{UA model}\label{ua-model}}

\hypertarget{ba-model}{%
\subsection{BA model}\label{ba-model}}

\hypertarget{fitting-model-to-real-data}{%
\section{Fitting model to real data}\label{fitting-model-to-real-data}}

\hypertarget{gpa-analyses}{%
\section{GPA analyses}\label{gpa-analyses}}

\hypertarget{conclusion-and-a-conjecture}{%
\section{Conclusion and a
Conjecture}\label{conclusion-and-a-conjecture}}

\hypertarget{next-steps}{%
\chapter{Next Steps}\label{next-steps}}

\newpage{}

\hypertarget{references}{%
\chapter*{References}\label{references}}
\addcontentsline{toc}{chapter}{References}

\hypertarget{refs}{}
\begin{CSLReferences}{1}{0}
\leavevmode\vadjust pre{\hypertarget{ref-Barabasi99}{}}%
Barabási, Albert-László, and Réka Albert. 1999. {``Emergence of Scaling
in Random Networks.''} \emph{Science} 286 (5439): 509--12.
\url{https://doi.org/10.1126/science.286.5439.509}.

\leavevmode\vadjust pre{\hypertarget{ref-shimura12}{}}%
Shimura, Takaaki. 2012. {``Discretization of Distributions in the
Maximum Domain of Attraction.''} \emph{Extremes} 15 (September): 1--19.
\url{https://doi.org/10.1007/s10687-011-0137-7}.

\end{CSLReferences}



\end{document}
